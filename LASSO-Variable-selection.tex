% Options for packages loaded elsewhere
\PassOptionsToPackage{unicode}{hyperref}
\PassOptionsToPackage{hyphens}{url}
%
\documentclass[
]{article}
\usepackage{amsmath,amssymb}
\usepackage{lmodern}
\usepackage{iftex}
\ifPDFTeX
  \usepackage[T1]{fontenc}
  \usepackage[utf8]{inputenc}
  \usepackage{textcomp} % provide euro and other symbols
\else % if luatex or xetex
  \usepackage{unicode-math}
  \defaultfontfeatures{Scale=MatchLowercase}
  \defaultfontfeatures[\rmfamily]{Ligatures=TeX,Scale=1}
\fi
% Use upquote if available, for straight quotes in verbatim environments
\IfFileExists{upquote.sty}{\usepackage{upquote}}{}
\IfFileExists{microtype.sty}{% use microtype if available
  \usepackage[]{microtype}
  \UseMicrotypeSet[protrusion]{basicmath} % disable protrusion for tt fonts
}{}
\makeatletter
\@ifundefined{KOMAClassName}{% if non-KOMA class
  \IfFileExists{parskip.sty}{%
    \usepackage{parskip}
  }{% else
    \setlength{\parindent}{0pt}
    \setlength{\parskip}{6pt plus 2pt minus 1pt}}
}{% if KOMA class
  \KOMAoptions{parskip=half}}
\makeatother
\usepackage{xcolor}
\IfFileExists{xurl.sty}{\usepackage{xurl}}{} % add URL line breaks if available
\IfFileExists{bookmark.sty}{\usepackage{bookmark}}{\usepackage{hyperref}}
\hypersetup{
  pdftitle={LASSO Variable selection},
  hidelinks,
  pdfcreator={LaTeX via pandoc}}
\urlstyle{same} % disable monospaced font for URLs
\usepackage[margin=1in]{geometry}
\usepackage{color}
\usepackage{fancyvrb}
\newcommand{\VerbBar}{|}
\newcommand{\VERB}{\Verb[commandchars=\\\{\}]}
\DefineVerbatimEnvironment{Highlighting}{Verbatim}{commandchars=\\\{\}}
% Add ',fontsize=\small' for more characters per line
\usepackage{framed}
\definecolor{shadecolor}{RGB}{248,248,248}
\newenvironment{Shaded}{\begin{snugshade}}{\end{snugshade}}
\newcommand{\AlertTok}[1]{\textcolor[rgb]{0.94,0.16,0.16}{#1}}
\newcommand{\AnnotationTok}[1]{\textcolor[rgb]{0.56,0.35,0.01}{\textbf{\textit{#1}}}}
\newcommand{\AttributeTok}[1]{\textcolor[rgb]{0.77,0.63,0.00}{#1}}
\newcommand{\BaseNTok}[1]{\textcolor[rgb]{0.00,0.00,0.81}{#1}}
\newcommand{\BuiltInTok}[1]{#1}
\newcommand{\CharTok}[1]{\textcolor[rgb]{0.31,0.60,0.02}{#1}}
\newcommand{\CommentTok}[1]{\textcolor[rgb]{0.56,0.35,0.01}{\textit{#1}}}
\newcommand{\CommentVarTok}[1]{\textcolor[rgb]{0.56,0.35,0.01}{\textbf{\textit{#1}}}}
\newcommand{\ConstantTok}[1]{\textcolor[rgb]{0.00,0.00,0.00}{#1}}
\newcommand{\ControlFlowTok}[1]{\textcolor[rgb]{0.13,0.29,0.53}{\textbf{#1}}}
\newcommand{\DataTypeTok}[1]{\textcolor[rgb]{0.13,0.29,0.53}{#1}}
\newcommand{\DecValTok}[1]{\textcolor[rgb]{0.00,0.00,0.81}{#1}}
\newcommand{\DocumentationTok}[1]{\textcolor[rgb]{0.56,0.35,0.01}{\textbf{\textit{#1}}}}
\newcommand{\ErrorTok}[1]{\textcolor[rgb]{0.64,0.00,0.00}{\textbf{#1}}}
\newcommand{\ExtensionTok}[1]{#1}
\newcommand{\FloatTok}[1]{\textcolor[rgb]{0.00,0.00,0.81}{#1}}
\newcommand{\FunctionTok}[1]{\textcolor[rgb]{0.00,0.00,0.00}{#1}}
\newcommand{\ImportTok}[1]{#1}
\newcommand{\InformationTok}[1]{\textcolor[rgb]{0.56,0.35,0.01}{\textbf{\textit{#1}}}}
\newcommand{\KeywordTok}[1]{\textcolor[rgb]{0.13,0.29,0.53}{\textbf{#1}}}
\newcommand{\NormalTok}[1]{#1}
\newcommand{\OperatorTok}[1]{\textcolor[rgb]{0.81,0.36,0.00}{\textbf{#1}}}
\newcommand{\OtherTok}[1]{\textcolor[rgb]{0.56,0.35,0.01}{#1}}
\newcommand{\PreprocessorTok}[1]{\textcolor[rgb]{0.56,0.35,0.01}{\textit{#1}}}
\newcommand{\RegionMarkerTok}[1]{#1}
\newcommand{\SpecialCharTok}[1]{\textcolor[rgb]{0.00,0.00,0.00}{#1}}
\newcommand{\SpecialStringTok}[1]{\textcolor[rgb]{0.31,0.60,0.02}{#1}}
\newcommand{\StringTok}[1]{\textcolor[rgb]{0.31,0.60,0.02}{#1}}
\newcommand{\VariableTok}[1]{\textcolor[rgb]{0.00,0.00,0.00}{#1}}
\newcommand{\VerbatimStringTok}[1]{\textcolor[rgb]{0.31,0.60,0.02}{#1}}
\newcommand{\WarningTok}[1]{\textcolor[rgb]{0.56,0.35,0.01}{\textbf{\textit{#1}}}}
\usepackage{graphicx}
\makeatletter
\def\maxwidth{\ifdim\Gin@nat@width>\linewidth\linewidth\else\Gin@nat@width\fi}
\def\maxheight{\ifdim\Gin@nat@height>\textheight\textheight\else\Gin@nat@height\fi}
\makeatother
% Scale images if necessary, so that they will not overflow the page
% margins by default, and it is still possible to overwrite the defaults
% using explicit options in \includegraphics[width, height, ...]{}
\setkeys{Gin}{width=\maxwidth,height=\maxheight,keepaspectratio}
% Set default figure placement to htbp
\makeatletter
\def\fps@figure{htbp}
\makeatother
\setlength{\emergencystretch}{3em} % prevent overfull lines
\providecommand{\tightlist}{%
  \setlength{\itemsep}{0pt}\setlength{\parskip}{0pt}}
\setcounter{secnumdepth}{-\maxdimen} % remove section numbering
\ifLuaTeX
  \usepackage{selnolig}  % disable illegal ligatures
\fi

\title{LASSO Variable selection}
\author{}
\date{\vspace{-2.5em}}

\begin{document}
\maketitle

\begin{Shaded}
\begin{Highlighting}[]
\CommentTok{\#preparation of data}
\FunctionTok{library}\NormalTok{(glmnet)}
\end{Highlighting}
\end{Shaded}

\begin{verbatim}
## Loading required package: Matrix
\end{verbatim}

\begin{verbatim}
## Loaded glmnet 4.1-4
\end{verbatim}

\begin{Shaded}
\begin{Highlighting}[]
\FunctionTok{library}\NormalTok{(readr)}
\NormalTok{df }\OtherTok{\textless{}{-}} \FunctionTok{read\_csv}\NormalTok{(}\StringTok{"bole\_layerfeatures .csv"}\NormalTok{)}
\end{Highlighting}
\end{Shaded}

\begin{verbatim}
## Rows: 95 Columns: 74
\end{verbatim}

\begin{verbatim}
## -- Column specification --------------------------------------------------------
## Delimiter: ","
## chr  (1): Borehole
## dbl (73): Elevation(m) -AHD, Lat, Lon, GW-AHD(12/06/22), Total_Thickness
(m)...
## 
## i Use `spec()` to retrieve the full column specification for this data.
## i Specify the column types or set `show_col_types = FALSE` to quiet this message.
\end{verbatim}

\begin{Shaded}
\begin{Highlighting}[]
\NormalTok{df[}\FunctionTok{is.na}\NormalTok{(df)] }\OtherTok{=} \DecValTok{0}
\NormalTok{df}\SpecialCharTok{$}\NormalTok{Lat}\OtherTok{\textless{}{-}} \SpecialCharTok{{-}}\NormalTok{df}\SpecialCharTok{$}\NormalTok{Lat}
\NormalTok{features }\OtherTok{\textless{}{-}}\NormalTok{ df[,}\SpecialCharTok{{-}}\DecValTok{1}\NormalTok{]}
\NormalTok{features }\OtherTok{\textless{}{-}}\NormalTok{ features[,}\SpecialCharTok{{-}}\DecValTok{4}\NormalTok{]}
\CommentTok{\# normalisation}
\NormalTok{features }\OtherTok{\textless{}{-}} \FunctionTok{scale}\NormalTok{(features)}
\end{Highlighting}
\end{Shaded}

To determine what value to use for lambda, we'll perform k-fold
cross-validation and identify the lambda value that produces the lowest
test mean squared error (MSE).

\begin{Shaded}
\begin{Highlighting}[]
\CommentTok{\#perform k{-}fold cross{-}validation to find optimal lambda value}
\CommentTok{\# Set alpha=1 in Lasso}
\NormalTok{cv\_model }\OtherTok{\textless{}{-}} \FunctionTok{cv.glmnet}\NormalTok{(}\FunctionTok{as.matrix}\NormalTok{(features[,}\DecValTok{1}\SpecialCharTok{:}\DecValTok{8}\NormalTok{]),}\FunctionTok{scale}\NormalTok{(df}\SpecialCharTok{$}\StringTok{\textasciigrave{}}\AttributeTok{GW{-}AHD(12/06/22)}\StringTok{\textasciigrave{}}\NormalTok{), }\AttributeTok{alpha =} \DecValTok{1}\NormalTok{)}
\CommentTok{\#find optimal lambda value that minimizes test MSE}
\NormalTok{best\_lambda }\OtherTok{\textless{}{-}}\NormalTok{ cv\_model}\SpecialCharTok{$}\NormalTok{lambda.min}
\NormalTok{best\_lambda}
\end{Highlighting}
\end{Shaded}

\begin{verbatim}
## [1] 0.1142405
\end{verbatim}

\begin{Shaded}
\begin{Highlighting}[]
\CommentTok{\#find coefficients of best model}

\NormalTok{best\_model }\OtherTok{\textless{}{-}} \FunctionTok{glmnet}\NormalTok{(}\FunctionTok{as.matrix}\NormalTok{(features[,}\DecValTok{1}\SpecialCharTok{:}\DecValTok{8}\NormalTok{]),}\FunctionTok{scale}\NormalTok{(df}\SpecialCharTok{$}\StringTok{\textasciigrave{}}\AttributeTok{GW{-}AHD(12/06/22)}\StringTok{\textasciigrave{}}\NormalTok{), }\AttributeTok{alpha =} \DecValTok{1}\NormalTok{, }\AttributeTok{lambda =}\NormalTok{ best\_lambda)}
\FunctionTok{coef}\NormalTok{(best\_model)}
\end{Highlighting}
\end{Shaded}

\begin{verbatim}
## 9 x 1 sparse Matrix of class "dgCMatrix"
##                                     s0
## (Intercept)               2.875772e-15
## Elevation(m) -AHD         6.553204e-02
## Lat                      -8.723143e-02
## Lon                       1.110378e-01
## Total_Thickness\u2028(m) -4.966538e-02
## num_Lithology_type        .           
## total_layer               .           
## nl_soil                   1.204036e-01
## tn_soil                   .
\end{verbatim}

Hence, the variables that effect on water level variations are the
Elevation(m)-AHD, Latitude, Longitude and number of Layers in soil.

\begin{Shaded}
\begin{Highlighting}[]
\CommentTok{\# obtain the results by plot}
\NormalTok{glmmod }\OtherTok{\textless{}{-}} \FunctionTok{glmnet}\NormalTok{(}\FunctionTok{as.matrix}\NormalTok{(features[,}\DecValTok{1}\SpecialCharTok{:}\DecValTok{8}\NormalTok{]),}\FunctionTok{scale}\NormalTok{(df}\SpecialCharTok{$}\StringTok{\textasciigrave{}}\AttributeTok{GW{-}AHD(12/06/22)}\StringTok{\textasciigrave{}}\NormalTok{), }\AttributeTok{alpha=}\DecValTok{1}\NormalTok{)}

\NormalTok{plot1}\OtherTok{\textless{}{-}}\FunctionTok{plot}\NormalTok{(glmmod, }\AttributeTok{xvar=}\StringTok{"lambda"}\NormalTok{,}\AttributeTok{label=}\ConstantTok{TRUE}\NormalTok{)}\SpecialCharTok{+}\FunctionTok{abline}\NormalTok{(}\AttributeTok{v=}\FunctionTok{log}\NormalTok{(best\_lambda), }\AttributeTok{col=}\StringTok{"red"}\NormalTok{)}
\end{Highlighting}
\end{Shaded}

\includegraphics{LASSO-Variable-selection_files/figure-latex/unnamed-chunk-4-1.pdf}

\end{document}
